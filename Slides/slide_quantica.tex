\documentclass[aspectratio=169]{beamer}
\usepackage[utf8]{inputenc}
\usepackage[T1]{fontenc}
\usepackage[brazil]{babel}
\usepackage{lmodern}
\usepackage{braket}

% --- USE O TEMA QUANTUM ---
\usetheme{Quantum}

% --- INFORMAÇÕES DA APRESENTAÇÃO ---
\title{Otimização de funções em módulo 2 utilizando DQI}
\subtitle{Algoritmo DQI max-XORSAT}
\author{Daniel Nocito \and Diogo Vieira \and Rafael Campos}
\institute{UFRJ}
\date{02 de Julho de 2025}

\begin{document}

% --- PÁGINA DE TÍTULO ---
\begin{frame}
  \titlepage
\end{frame}

% --- SLIDE DE ROTEIRO ---
\begin{frame}
  \frametitle{Roteiro}
  \tableofcontents
\end{frame}

\section{Entendimento do Problema}
\begin{frame}
  \frametitle{O Problema max-XORSAT}

  O é max-XORSAT? Traduzindo, significa :
  \begin{center}
    \textit{Maximizar a Satisfabilidade do XOR.}
  \end{center}

  Ou seja, encontrar um string binário (ou vetor) que satisfaça o maior número possível de equações lineares em $\mod 2$.

\end{frame}

% --- SLIDE DEFINIÇÃO DA FUNÇÃO OBJETIVO ---
\section{Definição da Função Objetivo}
\begin{frame}
  \frametitle{Definição da Função Objetivo}
  \begin{itemize}
    \item Como maximizar o número de equações satisfeitas?
  \end{itemize}
  Ora, um sistema linear em módulo 2 é da forma:
  \[
  Ax=b \mod2
  \longleftrightarrow
    \begin{cases} 
      a_{11} x_1 + a_{12} x_2 + \cdots + a_{1n}x_n = b_1 & \text{(mod 2)} \\
      a_{21} x_1 + a_{22} x_2 + \cdots + a_{2n}x_n = b_2 & \text{(mod 2)} \\
      \vdots \\
      a_{m1} x_1 + a_{m2} x_2 + \cdots + a_{mn}x_n = b_m & \text{(mod 2)}
    \end{cases}
    \]
  
  Para a função objetivo, definimos o resultado intermediário para cada linha como
  $$
  \begin{cases}
    1, & \text{se } a_i\cdot x = b_i \\
    -1, & \text{se } a_i\cdot x \neq b_i \\
  \end{cases}
  $$
\end{frame}

\begin{frame}
  \frametitle{Definição da Função Objetivo}
  Voltando ao problema max-XORSAT, a função objetivo $f(x)$ é, então, o somatório dos resultados intemediários:
  \[
    f(x)=\sum_{i=1}^m (-1)^{a_i \cdot x + b_i}
  \]

  \vfill
  \begin{block}{Ideia Geral da Func. Objetivo}
    Essa função objetivo conta o número de equações cuja escolha de $x$ satisfaz e subtrai o número de equações que não satisfazem.
  \end{block}
\end{frame}

% --- SLIDE DEFINIÇÃO DO PROBLEMA ---
\section{Definição do Problema}
\begin{frame}
  \frametitle{Definição do Problema}
    \textbf{Entrada:} Um sistema linear na forma de:
    \begin{itemize}
      \item Uma matriz $A \in \mathbb{F}_2^{m \times n}$; e
      \item Um vetor $b \in \mathbb{F}_2^m$ com $m > n$.
    \end{itemize}
    \vfill
    \textbf{Saída:} Um vetor $x \in \mathbb{F}_2^n$ que melhor maximiza $f$, definida adiante.
  \vfill
  \begin{block}{O Problema max-XORSAT}
  \begin{itemize} 
    \item \textbf{Objetivo:} Encontrar x que \textit{melhor soluciona} a equação $Ax=b \mod2$.
    \item \textbf{Metodologia}
    \begin{itemize}
        \item Definir uma função objetivo $f(x)$ para \textit{melhor soluciona}. \checkmark
        \item Aplicar o Algoritmo de Interferometria Quântica Decodificada (DQI).
    \end{itemize}
  \end{itemize}
  \end{block}
\end{frame}

\subsection{Por que usar o DQI?}
\begin{frame}
  \frametitle{Por que usar o DQI?}
  O problema Max-XORSAT é classificado como \textbf{NP-difícil}!

  \vfill
  \begin{block}{Vantagens do DQI}
    \begin{itemize}
      \item O DQI foca em encontrar soluções \textit{aproximadas}.
      \item A estratégia é reduzir uum problema de ptimização a um problema de decodificação usando a \textit{Transformada de Fourier Quântica}.
    \end{itemize}
  \end{block}
\end{frame}


\section{Algoritmo DQI}
% --- SLIDE ALGORITMO DQI ---
\begin{frame}
    \frametitle{Algoritmo DQI}

    \textbf{Passos do Algoritmo de Intereformetria Quântica Decodificada (DQI):}
    \vfill

    \begin{enumerate}
        \item Preparação do Estado Inicial
            \begin{itemize}
                \item Criação de Superposição Uniforme
            \end{itemize}
        \vfill
        \item Codificação da Função Objetivo na Fase
            \begin{itemize}
                \item Aplicar efeito da Função Objetivo
            \end{itemize}
        \vfill
        \item Aplicação da Transformada Quântica de Fourier (QFT)
            \begin{itemize}
                \item Causa interferência construtiva para favorecer as soluções desejadas
            \end{itemize}
        \vfill
        \item Medição e Decodificação Clássica
            \begin{itemize}
                \item Leitura da Superposição e interpretação clássica
            \end{itemize}
    \end{enumerate}
\end{frame}

\subsection{Circuito do DQI}
\begin{frame}
  \frametitle{Algoritmo DQI}

  \textbf{Circuito Quântico:}

  TODO: Adicionar figura do circuito.
  % \begin{figure}
    
  % \end{figure}
\end{frame}

% --- SLIDE PREPARAÇÃO DO ESTADO INICIAL ---
\subsection{Preparação do Estado Inicial}

\begin{frame}
  \frametitle{Preparação do Estado Inicial}

  Criação de estado de \textbf{Superposição Uniforme} de todos os $2^n$ estados possíveis para o vetor $x$ de variáveis.
  Isso é feito aplicando Hadamard em todos os registros.

  \vfill

  \begin{block}{Resultado da 1ª Etapa}
  Aplicando Hadamard a $n$ qubits no estado $\ket{0}$, resulta em:
  \[
    \ket{\psi_0}=\frac{1}{2^n} \sum_{x\in\{0,1\}^n}\ket{x}
  \]
  \end{block}
\end{frame}

% --- SLIDE CODIFICAÇÂO DA FUNÇÃO OBJETIVO NA FASE ---
\subsection{Codificação da Função Objetivo na Fase}

\begin{frame}
  Nessa etapa, a função objetivo é incorporada na \textit{fase} estado quântico. 

  \vfill

  Isso é feito através de um polinômio $P(f(x))$ da função objetivo, em que $P(f(x))\ket{x}$ é proporcional a amplitude do estado.

  \frametitle{Codificação da Função Objetivo na Fase}
  \[
      \ket{\psi_P}=N \sum_{x\in\{0,1\}^n}P(f(x))\ket{x}
  \]
  \begin{block}{Transformação com o Polinômio Objetivo}
      \begin{itemize}
          \item $N$ é a constante de normalização; e
          \item $P(f(x))$ é o polinômio da função objetivo.
      \end{itemize}
  \end{block}

  TODO: explicar o polinômio da Func obj
\end{frame}

\subsection{Tranformada Quântica de Fourier}
\begin{frame}
  \frametitle{Tranformada Quântica de Fourier}

  Nessa etapa, aplicamos a QFT em $|\psi_P\rangle$. 
  \vfill
  \textit{Fisicamente}, a QFT causa uma interferência construtiva nas amplitudes dos estados que correspondem às características importantes da função objetivo. 
  Nossa função objetivo possui seu espectro de Fourier \textbf{esparso} (i.e., a informação está concentrada em poucos coeficientes de Fourier: $N$ ou $-N$). TODO: confirmar se N*P(f(x)) = N ou -N
  
  \begin{block}{Relembrando a QFT}
    A QFT de $n$-qubits atua em um estado da base $|\mathbf{x}\rangle$ da seguinte forma:
    $$
    \text{QFT}|\mathbf{x}\rangle = \frac{1}{\sqrt{2^n}} \sum_{\mathbf{y} \in \{0,1\}^n} (-1)^{\mathbf{x} \cdot \mathbf{y}} |\mathbf{y}\rangle
    $$
  \end{block}
\end{frame}

\subsection{Medição}
\begin{frame}
  \frametitle{Medição}

  Após a aplicação da QFT, é aplicada a \textbf{medição}! 
  \vfill
  Ela resulta em uma string de bits $y$, que não é a solução $x$. 
  Na verdade, ela é a \textit{síndrome} de um código de correção de erros de paridade de baixa densidade (LDPC).
  TODO: porque?
  
  \vfill
  \begin{block}{Códigos de Verificação de Paridade de Baixa Densidade (LDPC)}
    Classe de Códigos de Correção de Erros lineares definidos por uma matriz de checagem de paridade que contém poucos 1s em relação aos 0s.
  \end{block}

  \begin{block}{Síndrome}
    Significa dizer que os bits de $y$ contém informações sobre as relações entre os bits da solução ótima $x$. Indicam quais verificações de paridade falharam.
  \end{block}
\end{frame}

\subsection{Decodificação Clássica}
\begin{frame}
  \frametitle{Decodificação Clássica}

  A última etapa consiste utilizar de um algoritmo de decodificação clássica com os parâmetros $y$, a síndrome, e $f$, a função objetivo para inferir a solução ótima (ou \textit{quase ótima}) $x$.

  \vfill

  Usamos o TODO:

\end{frame}

\section{DQI em Ação}
\begin{frame}
  \frametitle{DQI em Ação}

  TODO: adicionar o gif
  
\end{frame}

\section{Considerações Finais}
\begin{frame}
  \frametitle{Considerações Finais}
  
\end{frame}

\end{document}