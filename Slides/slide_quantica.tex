\documentclass[aspectratio=169]{beamer}
\usepackage[utf8]{inputenc}
\usepackage[T1]{fontenc}
\usepackage[brazil]{babel}
\usepackage{lmodern}
\usepackage{braket}

% --- USE O TEMA QUANTUM ---
\usetheme{Quantum}

% --- INFORMAÇÕES DA APRESENTAÇÃO ---
\title{Otimização de funções em módulo 2 utilizando DQI}
\subtitle{Algoritmo DQI max-XORSAT}
\author{Daniel Nocito \and Diogo Vieira \and Rafael Campos}
\institute{UFRJ}
\date{02 de Julho de 2025}

\begin{document}

% --- PÁGINA DE TÍTULO ---
\begin{frame}
  \titlepage
\end{frame}

% --- SLIDE DE ROTEIRO ---
\begin{frame}
  \frametitle{Roteiro}
  \tableofcontents
\end{frame}

\section{Definição do Problema}

% --- SLIDE DEFINIÇÃO DO PROBLEMA ---
\begin{frame}
  \frametitle{Definição do Problema}
  \begin{center}
      \textbf{Resolução de Sistemas em mod 2}
  \end{center}
  \[
    \begin{cases} 
      a_{11} x_1 + a_{12} x_2 + \cdots + a_{1n}x_n = b_1 & \text{(mod 2)} \\
      a_{21} x_1 + a_{22} x_2 + \cdots + a_{2n}x_n = b_2 & \text{(mod 2)} \\
      \vdots \\
      a_{m1} x_1 + a_{m2} x_2 + \cdots + a_{mn}x_n = b_m & \text{(mod 2)}
    \end{cases}
    \]
  \begin{itemize} 
    \item \textbf{Objetivo:} Encontrar x que melhor soluciona a equação $Ax=b$ (mod2)
    \item \textbf{Metodologia}
    \begin{itemize}
        \item Definir uma função objetivo $f(x)$ que favorece soluções que satisfazem mais equações
        \item Aplicar o Algoritmo DQI
    \end{itemize}
  \end{itemize}
\end{frame}

\section{Definição da Função Objetivo}

% --- SLIDE DEFINIÇÃO DA FUNÇÃO OBJETIVO ---
\begin{frame}
  \frametitle{Definição da Função Objetivo}
  
  Para cada equação $a_i \cdot x = b_i$, associar um valor de +1 para as soluções que satisfazem a equação e -1 para as que falham.

  \vspace{1cm}
  
  \begin{block}{Maximizar a função é equivalente a encontrar a melhor solução aproximada}
  \[
    f(x)=\sum_{i=1}^m (-1)^{a_i \cdot x + b_i}
  \]
  \end{block}
\end{frame}

\section{Algoritmo DQI}

% --- SLIDE ALGORITMO DQI ---

\begin{frame}
    \frametitle{Algoritmo DQI}

    \textbf{Passos do Algoritmo de Intereformetria Quântica Decodificada (DQI)}

    \vspace{2cm}

    \begin{enumerate}
        \item Preparação do Estado Inicial
            \begin{itemize}
                \item Criação de Superposição Uniforme
            \end{itemize}
        \item Codificação da Função Objetivo na Fase
            \begin{itemize}
                \item Aplicar efeito da Função Objetivo
            \end{itemize}
        \item Aplicação da Transformada Quântica de Fourier (QFT)
            \begin{itemize}
                \item Causa interferência construtiva para favorecer as soluções desejadas
            \end{itemize}
        \item Medição e Decodificação Clássica
            \begin{itemize}
                \item Leitura da Superposição e interpretação clássica
            \end{itemize}
    \end{enumerate}
\end{frame}

% --- SLIDE PREPARAÇÃO DO ESTADO INICIAL ---

\section{Preparação do Estado Inicial}

\begin{frame}
    \frametitle{Preparação do Estado Inicial}

    \textbf{Criação de Estado de Superposição Uniforme de todos os $2^n$ Estados possíveis para o vetor de variáveis x}

    \vspace{2cm}

    Aplicando Hadamard a $n$ qubits no estado $\ket{0}$, resulta em:

    \[
        \ket{\psi_0}=\frac{1}{2^n} \sum_{x\in\{0,1\}^n}\ket{x}
    \]
\end{frame}

% --- SLIDE CODIFICAÇÂO DA FUNÇÃO OBJETIVO NA FASE ---

\section{Codificação da Função Objetivo na Fase}

\begin{frame}
    \frametitle{Codificação da Função Objetivo na Fase}

    \[
        \ket{\psi_P}=N \sum_{x\in\{0,1\}^n}P(f(x))\ket{x}
    \]

    \begin{block}{Transformação com o Polinômio Objetivo}
        \begin{itemize}
            \item \textbf{N:} Constante de Normalização
            \item \textbf{P(f(x)):} Polinômio da Função Objetivo
        \end{itemize}
    \end{block}
\end{frame}






\end{document}